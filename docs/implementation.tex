\documentclass[a4paper, 11pt]{article}

% add warnings
\RequirePackage[l2tabu, orthodox]{nag}
\usepackage[all, warning]{onlyamsmath}

%codification of the document
\usepackage[utf8]{inputenc}
% font
\usepackage{newtxtext}  % Times and Helvetica
% mathematical features
\usepackage{amsmath, amssymb}
\usepackage{amsthm}
\theoremstyle{definition}
\usepackage{mathtools}
% screen
\usepackage{ascmac}
% url
\usepackage{url}
% for images
\usepackage{graphicx}
% chemical formula
\usepackage[version=3]{mhchem}
% appendix
\usepackage[toc,page]{appendix}
% add bibliography in TOC
\usepackage[nottoc]{tocbibind}
% hyperlink
\usepackage{hyperref}
% color
\usepackage{color}
% page size and margins
\usepackage{geometry}
% citation
\usepackage{cite}

\usepackage{bm}

% *matter for article class
% https://tex.stackexchange.com/questions/154646/is-there-an-easy-way-to-get-the-frontmatter-mainmatter-and-backmatter-in-a-l
\makeatletter
\newcommand\frontmatter{%
  \cleardoublepage
  %\@mainmatterfalse
  \pagenumbering{roman}}
\newcommand\mainmatter{%
  \cleardoublepage
  % \@mainmattertrue
  \pagenumbering{arabic}}
\makeatother

%%%%%%%%%%%%%%%%%%%%%%%%%%%%%%%%%%%%%%%%%%%%%%%%%%%%%%%%%%%%%%%%%%%%%%%%%%%%%%%%
% Macros
%%%%%%%%%%%%%%%%%%%%%%%%%%%%%%%%%%%%%%%%%%%%%%%%%%%%%%%%%%%%%%%%%%%%%%%%%%%%%%%%
% follow https://cdn.journals.aps.org/files/styleguide-pr.pdf
\newcommand{\secref}[1]{Section~\ref{#1}}
\newcommand{\appendixref}[1]{Appendix~\ref{#1}}
\newcommand{\software}[1]{\texttt{#1}}

\newcommand{\term}[1]{\textit{#1}}
\newcommand{\todo}[1]{\textcolor{red}{TODO: #1}}
\newcommand{\norm}[1]{\left\lVert#1\right\rVert}
% https://zrbabbler.hatenablog.com/entry/20120411/1334151482
\newcommand{\relmiddle}[1]{\mathrel{}\middle#1\mathrel{}}
\newcommand{\set}[2]{\left\{ #1 \relmiddle| #2 \right\}}
\newcommand{\Z}{\mathbb{Z}}
\newcommand{\R}{\mathbb{R}}

\newtheorem{theorem}{Theorem}[section]
\newtheorem{corollary}{Corollary}[theorem]
\newtheorem{lemma}[theorem]{Lemma}
\newtheorem{definition}[theorem]{Definition}

\title{Implementation Note on MLIP}
\author{Kohei Shinohara}
\date{\today}

\begin{document}
\maketitle

%%%%%%%%%%%%%%%%%%%%%%%%%%%%%%%%%%%%%%%%%%%%%%%%%%%%%%%%%%%%%%%%%%%%%%%%%%%%%%%
\section{Order Parameter}
Consider a neighboring atomic density of \term{atom} $i$ between \term{element} $s$, $\rho^{(i, s)}(\mathbf{r})$, and expand it with basis functions,
\begin{align}
  \rho^{(i, s)}(\mathbf{r})
    =: \sum_{n=1}^{n_{\mathrm{max}}} \sum_{l=0}^{l \leq l_{\mathrm{max}}} \sum_{m=-l}^{l} a^{(i, s)}_{nlm} f_{n}(r) Y_{lm}(\hat{\mathbf{r}}).
\end{align}
We refer the coefficients $a^{(i, s)}_{nlm}$ as \term{order parameters}.

Let $\mathcal{S}$ be a set of pair elements
\footnote{
  A square bracket $[ \cdot ]$ indicate an unordered set.
},
\begin{align}
  \mathcal{S} = \left\{ [A, A], [A, B], [B, B], \dots \right\}.
\end{align}
It is useful to generalize an order parameter with a set of pair elements, $[s, s'] \in \mathcal{S}$, as
\begin{align}
  a^{(i)}_{nlm, [s, s'] } \coloneqq
  \begin{cases}
    a^{(i, s'')}_{nlm} & (\exists s'' \,s.t.\, [s_{i}, s''] = [s, s'] ) \\
    0 & (\mathrm{otherwise})
  \end{cases},
\end{align}
where $s_{i}$ denotes a element type of atom $i$.

In practice, the order parameters are calculated with a neighbor list within a given cut off radius $r_{c}$,
\begin{align}
  a^{(i, s)}_{nlm} = \sum_{j \,s.t.\, r_{ij} \leq r_{c}, s_{j} = s} f_{n}(r_{ij}) Y_{lm}(\hat{\mathbf{r}_{ij}})^{\ast}.
\end{align}
Note that radial basis function $f_{n}(\cdot)$ is a real function.

%%%%%%%%%%%%%%%%%%%%%%%%%%%%%%%%%%%%%%%%%%%%%%%%%%%%%%%%%%%%%%%%%%%%%%%%%%%%%%%
\section{Structural Feature}
Consider $p$th-order polynomial of $\{ a^{(i)}_{nlm, t}\}_{n, l, m, t \in \mathcal{S}}$ which is invariant with $O(3)$ actions.

\term{angular-element pairs}
\begin{align}
  \mathcal{K}_{p}
    = \set{ [ (l_{1}, t_{1}), \dots, (l_{p}, t_{p}) ] }{ l_{1}, \dots, l_{p} = 0, \dots, l_{\mathrm{max}}, t_{1}, \dots, t_{p} \in \mathcal{S} }
\end{align}

\term{Structural feature}
\begin{align}
  d^{(i)}_{n, \mathbf{k}, \sigma_{\mathbf{k}}}
  \coloneqq \sum_{m_{1}=-l_{1}}^{l_{1}} \dots \sum_{m_{p}=-l_{p}}^{l_{p}} C^{l_{1} \dots l_{p}, \sigma_{\mathbf{k}} }_{m_{1} \dots m_{p}} a^{(i)}_{nl_{1}m_{1}, t_{1}} \dots a^{(i)}_{n l_{p} m_{p}, t_{p}}
  \quad \mbox{where} \quad \mathbf{k} = [(l_{1}, t_{1}), \dots, (l_{p}, t_{p}) ] \in \mathcal{K}_{p}.
\end{align}
The index $\sigma_{\mathbf{k}}$ distinguishes several Irreps with the same unordered set of angular numbers $[l_{1}, \dots, l_{p}]$.
We refer the tuple $(n, \mathbf{k}, \sigma_{k})$ as a \term{feature index}.

Regardless of element type of atom $i$, when $t_{1} \cap \dots \cap t_{p} = \varnothing$, the structural feature becomes zero.
Thus, we restrict the domain of $\mathcal{K}_{p}$ as \term{effective angular-element pairs},
\begin{align}
  \overline{\mathcal{K}}_{p}
  \coloneqq
  \set{ [(l_{1}, t_{1}), \dots, (l_{p}, t_{p}) ] \in \mathcal{K}_{p} }{ t_{1} \cap \dots \cap t_{p} \neq \varnothing }.
\end{align}

%%%%%%%%%%%%%%%%%%%%%%%%%%%%%%%%%%%%%%%%%%%%%%%%%%%%%%%%%%%%%%%%%%%%%%%%%%%%%%%
\section{Polynomial Feature}

\begin{align}
  \overline{\mathcal{K}} \coloneqq \overline{\mathcal{K}_{1}} \cup \overline{\mathcal{K}_{2}} \cup \dots
\end{align}
\todo{hyperparamter for $\overline{\mathcal{K}}$}

\term{elements intersection}
\begin{align}
  e(\mathbf{k}) \coloneqq t_{1} \cap \dots \cap t_{p} \quad \mbox{where} \quad \mathbf{k} = [(l_{1}, t_{1}), \dots, (l_{p}, t_{p})] \in \overline{\mathcal{K}}.
\end{align}

\term{effective feature indices families}
\begin{align}
  \overline{\mathcal{P}}_{q}
  \coloneqq \set{[(n_{1}, \mathbf{k}_{1}, \sigma_{\mathbf{k}_{1}}), \dots, (n_{q}, \mathbf{k}_{q}, \sigma_{\mathbf{k}_{q}})]}{ \mathbf{m}_{1}, \dots, \mathbf{k}_{q} \in \overline{\mathcal{K}}, e(\mathbf{k}_{1}) \cap \dots \cap e(\mathbf{k}_{q}) \neq \varnothing }
\end{align}

\term{polynomial feature}
\begin{align}
  d^{(i)}_{[(n_{1}, \mathbf{k}_{1}, \sigma_{\mathbf{k}_{1}}), \dots, (n_{q}, \mathbf{k}_{q}, \sigma_{\mathbf{k}_{q}})]}
  &\coloneqq d^{(i)}_{ n_{1} \mathbf{k}_{1} \sigma_{\mathbf{k}_{1}} } \dots d^{(i)}_{ n_{q} \mathbf{k}_{q} \sigma_{\mathbf{k}_{q}} } \nonumber \\
  &\mbox{where} \quad
    [(n_{1}, \mathbf{k}_{1}, \sigma_{\mathbf{k}_{1}}), \dots, (n_{q}, \mathbf{k}_{q}, \sigma_{\mathbf{k}_{q}})] \in \overline{\mathcal{P}}_{q}
\end{align}


%%%%%%%%%%%%%%%%%%%%%%%%%%%%%%%%%%%%%%%%%%%%%%%%%%%%%%%%%%%%%%%%%%%%%%%%%%%%%%%
\section{Potential Energy Model}

\begin{align}
  E(\{ \mathbf{r}_{i} \}) &= \sum_{i=1}^{N} F \left( \{ d^{(i)}_{ [\mathbf{f}_{1}, \dots, \mathbf{f}_{q}] } \}_{[\mathbf{f}_{1}, \dots, \mathbf{f}_{q}] \in \overline{\mathcal{P}}_{q}}^{q=1,\dots} \right) \\
  F \left( \{ d^{(i)}_{ [\mathbf{f}_{1}, \dots, \mathbf{f}_{q}] } \}_{[\mathbf{f}_{1}, \dots, \mathbf{f}_{q}] \in \overline{\mathcal{P}}_{q}}^{q=1,\dots} \right)
  &= \sum_{q \geq 1} \sum_{ [\mathbf{f}_{1}, \dots, \mathbf{f}_{q}] \in \overline{\mathcal{P}}_{q} } w_{ [\mathbf{f}_{1}, \dots, \mathbf{f}_{q}] } d^{(i)}_{ [\mathbf{f}_{1}, \dots, \mathbf{f}_{q}] }
\end{align}
Note that the weight $w_{ [\mathbf{f}_{1}, \dots, \mathbf{f}_{q}] }$ does not depend on an element type of atom $i$.

%%%%%%%%%%%%%%%%%%%%%%%%%%%%%%%%%%%%%%%%%%%%%%%%%%%%%%%%%%%%%%%%%%%%%%%%%%%%%%%
\section{Derivative}

\begin{align}
  \partial E
  &= \sum_{i} \sum_{q} \sum_{ [\mathbf{f}_{1}, \dots, \mathbf{f}_{q}] \in \overline{\mathcal{P}}_{q} }
        w_{ [\mathbf{f}_{1}, \dots, \mathbf{f}_{q}] } \partial d^{(i)}_{ [\mathbf{f}_{1}, \dots, \mathbf{f}_{q}] } \\
  \partial d^{(i)}_{ [\mathbf{f}_{1}, \dots, \mathbf{f}_{q}] }
  &= \left( \partial d^{(i)}_{\mathbf{f}_{1}} \right) d^{(i)}_{\mathbf{f}_{2}} \dots d^{(i)}_{\mathbf{f}_{q}} + \dots \\
  \partial d^{(i)}_{n, \mathbf{k}, \sigma_{\mathbf{k}}}
  &=
\end{align}

\todo{permutation symmery of $ C^{l_{1} \dots l_{p}, \sigma_{\mathbf{k}} }_{m_{1} \dots m_{p}} $}

%%%%%%%%%%%%%%%%%%%%%%%%%%%%%%%%%%%%%%%%%%%%%%%%%%%%%%%%%%%%%%%%%%%%%%%%%%%%%%%
% References
%%%%%%%%%%%%%%%%%%%%%%%%%%%%%%%%%%%%%%%%%%%%%%%%%%%%%%%%%%%%%%%%%%%%%%%%%%%%%%%
% \bibliographystyle{unsrt}
% \bibliography{reference}

\end{document}
